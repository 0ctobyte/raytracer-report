%*******************************************************
% Manual
%*******************************************************
\pdfbookmark[1]{Manual}{manual}
\chapter{Manual}

\section{Usage}
Running the ray tracer is very simple. Assuming the scene is specified by a file
named \verb|scene.lua|, then to run the ray tracer on the specified scene file
the following command shall be entered into the shell:
\begin{verbatim}
    ./rt scene.lua
\end{verbatim}
An image file will be produced in the same directory with the filename specified
in the  \verb|scene.lua| file. There will be no user interaction while the ray
tracer executable is running. The program will, however, display a progress bar.

There are no other command line options available for the executable. However,
optional parameters can be specified in the Lua scene file which will be
described in the next section.

\section{Modelling Language Extensions}
The LUA modelling language was extended to support many of the objectives that
were implemented for this project. Below is a summary of the changes made to old
commands as well as new commands that were added to the language for each of the
objectives implemented for this project.

\subsection*{Render Command}
The render command was extended to accept optional parameters to support
multi-threading, mirror reflections, anti-aliasing, soft shadows, and glossy
reflections and background images:
\begin{verbatim}
  gr.render(<scene_root_node>, <filename>, <width>, <height>, <eye>, <view>,
  <up>, <fov>, <ambient>, <lights>, <threads> = 1, <recurse_level> = 0,
  <aa_samples> = 1, <shadow_samples> = 1, <glossy_samples> = 1,
  <background_filename> = null)
\end{verbatim}
The \verb|<threads>| option is used to specify the number of threads used by the
ray tracer. The default is one thread.

The \verb|<recurse_level>| option specifies the maximum number of times that the
\verb|a4_trace_ray| function is called recursively. The default for this option
is zero in which case mirror reflections and refractions will not be rendered.

The \verb|<aa_samples>| option specifies the number of rays that are cast for each
pixel in order to perform anti-aliasing. The default is one ray per pixel.

The \verb|<shadow_samples>| option specifies the number of shadow rays that are
cast for each pixel in order to perform soft shadow rendering. The default is
one shadow ray per pixel.

The \verb|<glossy_samples>| option specifies the number of reflection rays that
are cast for each pixel in order to perform glossy reflection rendering. The
default is one reflection ray per pixel.

The \verb|<background_filename>| option specifies the filename of the image to
use as the background for the scene. The default is a null value meaning no
background image will be used.

\subsection*{Extra Primitives}
The following commands were added to the language in order to support rendering
of extra primitives.
\begin{itemize}
  \item \verb|gr.cone(<name>)|
  \item \verb|gr.cylinder(<name>)|
  \item \verb|gr.disc(<name>)|
  \item \verb|gr.plane(<name>)|
  \item \verb|gr.torus(<name>)|
\end{itemize}

\subsection*{Constructive Solid Geometry}
The following commands were added to the language in order to support rendering
of constructive solid geometry objects.
\begin{itemize}
  \item \verb|gr.csg_union(<name>, <geometry_node>, <geometry_node>)|
  \item \verb|gr.csg_intersection(<name>, <geometry_node>, <geometry_node>)|
  \item \verb|gr.csg_difference(<name>, <geometry_node>, <geometry_node>)|
\end{itemize}

\subsection*{Anti-Aliasing}
See the modifications made to \verb|gr.render| command above in order to support
anti-aliasing in the modelling language.

\subsection*{Soft Shadows}
The following command was added to the language to create area lights that are
used for the soft shadow computations:
\begin{verbatim}
  gr.disc_light(<position>, <colour>, <attenuation>, <normal>, <radius>)
\end{verbatim}
The command is a modification of the \verb|gr.light| command with two new
parameters added. The \verb|<normal>| parameter specifies the normal of the disc
light source and the \verb|<radius>| parameter specifies the radius of the disc
light source.

\subsection*{Texture Mapping}
The following command was added to the language to apply a texture image file
to the specified \verb|node|:
\begin{verbatim}
  node:set_texture(<filename>)
\end{verbatim}
The \verb|node| object must have had a material applied before using this
command via the \verb|set_material| command.

\subsection*{Bump Mapping}
The following command was added to the language to apply a bump map image file
to the specified \verb|node|:
\begin{verbatim}
  node:set_bumpmap(<filename>)
\end{verbatim}
The \verb|node| object must have had a material applied before using this
command via the \verb|set_material| command.

\subsection*{Phong Shading}
The following command is used to construct a mesh with a set of vertices and a
set of faces that contain indices into the vertex list:
\begin{verbatim}
  gr.tri_mesh(<name>, <verts>, <faces>)
\end{verbatim}
This command differs from the \verb|gr.mesh| command in that it subdivides the
faces into triangles as well as generating per vertex normals on instantiation.

\subsection*{Refraction}
The \verb|gr.material| command was extended as follows:
\begin{verbatim}
  gr.material(<diffuse>, <specular>, <shininess>, <ni> = 0)
\end{verbatim}
The optional \verb|<ni>| parameter was added to specify the index of refraction
of the material. The default value is zero which indicates that the material is
not refractive.

\subsection*{Glossy Reflection}
See the modifications made to the \verb|gr.render| command above in order to
support anti-aliasing.

\subsection*{Perlin Noise}
The following command was added to the language to apply a procedurally
generated texture using Perlin noise to the specified \verb|node|:
\begin{verbatim}
  node:set_perlin(<type>)
\end{verbatim}
The \verb|<type>| option specifies which procedural texture to apply to the
node. The \verb|node| object must have had a material applied before using this
command via the \verb|set_material| command.

