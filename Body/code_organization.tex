%*******************************************************
% Code Organization
%*******************************************************
\pdfbookmark[1]{Code Organization}{code organization}
\chapter{Code Organization}

\section{Directory Structure}
All code and data files are located in the \verb|~/cs488/handin/A5| directory.

A \verb|README| file is located in the above directory detailing how to build
and use the ray tracer. It also lists all the objectives implemented for the
project.

\verb|*.cpp| and \verb|*.hpp| source files are located in the \verb|src| 
subdirectory. The \verb|Makefile| is in this subdirectory as well.

The \verb|data| subdirectory contains all the LUA scripts used to demonstrate
the objectives implemented for the project. It also contains a \verb|textures|,
\verb|bumps|, \verb|backgrounds| and \verb|objs| subdirectories which contain
the texture images, the bump maps, the background images and the mesh
definitions respectively. The textures, bump maps and background images must be
in PNG format. The mesh definition files must be in the OBJ format.

\section{Code Map}
Below is a list of all the source files in the project and brief description of
their purpose.

\subsection*{a4.cpp, a4.hpp}
This is the where the core of the ray tracer is contained. This file includes
the recursive \verb|a4_trace_ray| function as well as functions to generate
reflection and refaction rays. It also contains code for thread creation and the
\verb|a4_lighting| function which applies the Lambertion lighting model to each
pixel.

\subsection*{algebra.cpp, algebra.hpp}
These files contains \verb|Point2D|, \verb|Point3D|, \verb|Vector2D|,
\verb|Vector3D|, \verb|Matrix4x4| and \verb|Colour| classes as well as 
structures to represent \verb|Rays| and \verb|Intersections|.

\subsection*{image.cpp, image.hpp}
These files contains a class to load, save and manipulate PNG image files.

\subsection*{light.cpp, light.hpp}
These files contains the \verb|Light| class to represent point lights and the
\verb|DiscLight| class to represent area lights.

\subsection*{lua488.hpp}
This file just includes some of the Lua headers.

\subsection*{main.cpp}
This is the main entry point of the program. It just initializes the Lua
environment and hands off execution to the Lua interpreter which begins
interpreting the given Lua scene script.

\subsection*{material.cpp, material.hpp}
These files contain the \verb|Material| and \verb|PhongMaterial| class which
represents a material. The \verb|PhongMaterial| class was extended in order to
support textures, bump maps and index of refraction attributes.

\subsection*{mesh.cpp, mesh.hpp}
These files contain the implementation of the \verb|Mesh| class which holds a
list of vertices and faces. The \verb|Mesh| class is a sub-class of the
\verb|Primitive| class which means it must implement the \verb|intersection|
method. A new class was added, \verb|TriMesh|, which is a sub-class of 
\verb|Mesh| which subdivides a mesh's face list into triangles and then
generates and stores per-vertex normals on instantiation.

\subsection*{perlin.cpp, perlin.hpp}
These files contain the \verb|Perlin| class which contains only static methods
that generate scaling factors (which follow a certain pattern depending on the
desired texture) that are used to modify the diffuse colour of a material.

\subsection*{polyroots.cpp, polyroots.hpp}
These files contain functions to solve quadratic, cubic and quartic polynomial
equations.

\subsection*{primitive.cpp, primitive.hpp}
These files contain a number of primitive classes described in the last section.
Listed below are the extra primitive classes that were added to this file:
\begin{itemize}
  \item \verb|Cone|
  \item \verb|Cylinder|
  \item \verb|Disc|
  \item \verb|Plane|
  \item \verb|Torus|
\end{itemize}
Each of these primitive classes implement the \verb|intersection| method.

\subsection*{scene.cpp, scene.hpp}
These files contain the \verb|SceneNode| class and it's sub-classes, most 
notably the \verb|GeometryNode| and \verb|ConstructiveSolidGeometryNode| 
classes. The \verb|SceneNode| class is used to construct a scene graph
containing a number of primitives and meshes.

\subsection*{scene\_lua.cpp, scene\_lua.hpp}
These files contain the implementation of the custom Lua commands used for the
modelling language. All new Lua commands and modifications of old commands were
added here.

