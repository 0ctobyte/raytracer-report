%*******************************************************
% Introduction
%*******************************************************
\pdfbookmark[1]{Introduction}{introduction}
\chapter{Introduction}

\section{Purpose}
The purpose of this project is to design and implement a functional ray tracer
satisfying the ten objectives listed below. The ten objectives are of reasonable
complexity and are functionalities which I believe are necessary for a 
reasonably complex ray tracer to posses.

\section{Statement}
I will attempt to build a reasonably complex ray tracer able to generate
photorealistic images of complex scenes that satisfy the ten objectives listed
below. I am also interested in implementing extra functionality such as parallel
ray tracing, uniform space subdivision, caustics using illumination map and
motion blur. I will generate scenes that display the functionality of each of
the objectives listed below and if time permits I will also attempt to satisfy
any extra objectives. As a note, I have implemented mirror reflections as part
of assignment four and so have not included it as an objective for ths project.

I find the topic of ray tracing to be very interesting to me and I have spent
considerable amount of time reading the seminal literature related to ray
tracing techniques. The challenge for me will be to implement these techniques
in an efficient manner while generating high quality images. Soft shadows,
glossy reflection and anti-aliasing (three of the techniques I wish to implement
for this project) are all techniques that increase the time it takes to produce
a ray traced image and it will be quite a challenge to implement these
techniques efficiently.

I hope that at the completion of this project I will learn many of the important
algorithms and techniques used in ray tracing as well as learning new and better
methods of efficient programming. I hope to also keep working on this project
even after completing this course and continue to implement other modern ray
tracing techniques.

\section{Objectives}
Below is a list of the objectives for this project in the order in which they
were completed. 

\begin{itemize}
  \item Extra Primitives
  \item Constructive Solid Geometry
  \item Anti-Aliasing
  \item Soft Shadows
  \item Texture Mapping
  \item Bump Mapping
  \item Phong Shading
  \item Refraction
  \item Glossy Reflection
  \item Perlin Noise
\end{itemize}

The two objectives I had stuggled with the most during the
project was bump mapping and perlin noise. Although the concept of bump mapping
is similar to that of texture mapping (which I had found relatively simple to
implement), I struggled with understangin the mathematics behind the
implementation of bump mapping. For Perlin noise, I was able to implement the
relatively straight-forward noise function but struggled to understand how to
use the noise function in a way to generate meaningful textures.
