%*******************************************************
% Constructive Solid Geometry
%*******************************************************
\section{Constructive Solid Geometry}

Constructive Solid Geometry was implemented by creating a
\newline \verb|ConstructiveSolidGeometryNode| class which sub-classes the 
\newline \verb|GeometryNode| class. The \verb|ConstructiveSolidGeometryNode| class 
contains references to two \verb|GeometryNode| which represents the two
\newline \verb|GeometryNode|'s to be combined. Note that this structure allows for a
hierarchy of \verb|ConstructiveSolidGeometryNode|'s.

Three sub-classes were derived from the \newline \verb|ConstructiveSolidGeometryNode|
class. These are the \verb|UnionNode|, \verb|IntersectionNode| and
\verb|DifferenceNode|. These nodes allow for the union, intersection and
difference of two \verb|GeometryNode|'s respectively.

Each of the three sub-classes implement the \verb|intersection| method. The
implementation of the \verb|intersection| methods are described below.

\subsection*{UnionNode}
A \verb|UnionNode| can be created by using the following Lua command: 
\begin{lstlisting}
    gr.union_node(<name>, <node_a>, <node_b>)
\end{lstlisting}

The \verb|UnionNode| is easily implemented by testing intersection on both of
the given \verb|GeometryNode|'s. If the ray intersects with either of the 
\verb|GeometryNode|'s then the \verb|intersection| method selects the closest
intersection point to the ray's origin (if both \verb|GeometryNode|'s have been 
intersected  and returns true. 

\subsection*{IntersectionNode}
An \verb|IntersectionNode| can be created by using the following Lua command:
\begin{lstlisting}
    gr.intersection_node(<name>, <node_a>, <node_b>)
\end{lstlisting}

The \verb|IntersectionNode| is easily implemented by testing intersection on
both of the given \verb|GeometryNode|'s. If the ray intersects both of the
\verb|GeometryNode|'s then the \verb|intersection| method selects the closest
intersection point and returns true.

\subsection*{DifferenceNode}
A \verb|DifferenceNode| can be created by using the following Lua command
\begin{lstlisting}
    gr.difference_node(<name>, <node_a>, <node_b>)
\end{lstlisting}

This node was the most difficult to implement since I had not extended the
\verb|Intersection| structure to contain the intersection intervals. The
algorithm is as follows. First, an intersection test is performed on 
\verb|node_a|. If the ray intersects \verb|node_a| then an intersection test is
performed on \verb|node_b|. If \verb|node_b| has not been intersected then the
method returns true with the intersection information of \verb|node_a| contained
in the \verb|Intersection| object. Otherwise, the closest intersection point of 
the two nodes are determined. If the intersection point of \verb|node_a| is 
closest then the method simply returns true with the intersection point of 
\verb|node_a|. If not, then another ray is cast starting at the intersection
point of the \verb|node_a| and again it is determined whether the intersection
of \verb|node_a| is the closest or not.

The idea behind this algorithm is to find an intersection point on \verb|node_a|
that is closer to the ray's origin than an intersection point on \verb|node_b|.

