%*******************************************************
% Extra Primitives
%*******************************************************
\section{Extra Primitives}

This section will described how the \verb|intersection| method was
implemented for each of the extra primitives. As mentioned in the previous
section, the primtive classes were added to the \verb|primitives.cpp| source 
file.

The \verb|intersection| method takes two parameters. The first parameter is a
\verb|Ray| object and the second parameter is an \verb|Intersection| object. 

A ray can be represented by the equation $R = O + t*D$ where $O$ is the ray's 
origin and $D$ is the ray's direction. The \verb|Ray| object contains the ray's 
origin and direction. The point, $R$, on the ray can be determined given the $t$ 
parameter. The intersection code aims to calculate the value of the $t$ 
parameter such that the point on ray's line is the point of intersection on the 
primitive. 

The \verb|Intersection| object is where the intersection details are
stored such as the point of intersection, the normal at the point of
intersection (used for lighting calculations), the material of the primitive
(used for lighting calculations), the texture coordinates (used to index the
texture image), and the tangent vectors (used in the calculation for bump
mapping). All intersection tests are done in the primtives' local coordinate
system with unit sized primitives.

\subsection*{Cone}
The \verb|gr.cone(<name>)| Lua is used to create a cone.

The cone primitive is specified by the implicit equation $x^2 + y^2 = z^2$ and 
unit height, $h$. The orientation of the cone is along the z-axis. The implicit 
equation specifies an infinite double ended cone along the z-axis. The finite 
single-ended cone is specified by the equation $x^2 + y^2 = z^2, -h < z < h$. 

In order to determine the intersection of a ray with a cone the implicit
equation must be solved by substituting in the equation of the ray:
\begin{equation}
  (O.x + t\times D.x)^2 + (O.y + t\times D.y)^2 = (O.z + t\times D.z)^2
  \label{cone1}
\end{equation}
This equation expands to:
\begin{equation}
\begin{split}
  (D.x^2 + D.y^2 - D.z^2)\times t^2 + \\
  (2O.x\times D.x + 2O.y\times D.y - 2O.z\times D.z)\times t \\
  + (O.x^2 + O.y^2 - O.z^2) = 0\label{cone2}
\end{split}
\end{equation}
The equation specified in~(\ref{cone2}) can be solved for $t$ using the
quadratic equation. If the value of $t$ cannot be determined or the if $t$ is
negative, indicating that the intersection point is behind the ray's origin,
then there is no intersection. If $t$ is positive then the ray intersects the
cone and the intersection point is checked to make sure that the inequality $-h
< z < h$ is satisfied.

The normal vector is calculated by the following equation:
\begin{equation}
  \vec{N} = \begin{bmatrix} 2Q.x & 2Q.y & -2Q.z
  \end{bmatrix}^{T}
\end{equation}
Where $Q$ is the intersection point. The equation is essentially calculating the
gradient of the intersection point over the surface of the cone.

The texture coordinates for the cone primitive are calculated as follows:
\begin{equation}
  u = \frac{acos(Q \cdot \begin{bmatrix} 1.0 & 0.0 & 0.0 \end{bmatrix}^{T})}
  {\pi}
\end{equation}
\begin{equation}
  v = \frac{Q}{h}
\end{equation}
Essentially, the $u$ coordinate is the angle of rotation of the vector (from
the intersection point to the origin) about the z-axis divided by $\pi$ in 
order to scale it to the range $[0, 1]$. The $v$ coordinate is calculated by
dividing the position of the intersection point along the z-axis by the height
of the cone.

\subsection*{Cylinder}
The \verb|gr.cylinder(<name>)| Lua command is used to create a cylinder.

The cylinder primitive is specified by the implicit equation $x^2 + y^2 = 1$ and
unit height, $h$. The orientation of the cylinder is along the z-axis. The
finite cylinder must satisfy the equation $\frac{-h}{2} < z < \frac{h}{2}$.

Similar to the cone, the intersection of a ray with a cylinder can be determined
by solving a quadratic equation with the following coefficients:
\begin{equation}
  A = D.x^2 + D.y^2 \\
  B = 2O.x\times D.x + 2O.y\times D.y \\
  C = O.x^2 + O.y^2 - 1
\end{equation}
The intersection of the ray with the end caps of the cylinder are determined by
testing intersection between the ray and a disc which is described later in this
section.

The normal vector is easily determined. It is essentially the vector from the
origin to the intersection point and dropping the $z$ coordinate.

\subsection*{Disc}
The \verb|gr.disc(<name>)| command is used to create a disc centered at the
origin and lying on the $xy$-plane with unit radius, $r$.

To determine the intersection of the ray with the disc, it is a simple matter of
determining if the ray intersects the plane that contains the disc and then
validating whether the intersection point lies within the disc's area. The
ray-plane intersection is described later in this section. To validate whether
the intersection point is within the disc's area, the following inequality must
be satisfied:
\begin{equation}
  Q\cdot Q \eq r^2
\end{equation}
Where $Q$ is the vector from the origin to the intersection vector.

\subsection*{Plane}

\subsection*{Torus}

