%*******************************************************
% Extra Primitives
%*******************************************************
\section{Extra Primitives}

This section will described how the \verb|intersection| method was
implemented for each of the extra primitives. As mentioned in the previous
section, the primtive classes were added to the \verb|primitives.cpp| source 
file.

The \verb|intersection| method takes two parameters. The first parameter is a
\verb|Ray| object. A ray can be represented by the equation $R = O + t*D$ where
$O$ is the ray's origin and $D$ is the ray's direction. The \verb|Ray| object
contains the ray's origin and direction. The point, $R$, on the ray can be 
determined given the $t$ parameter. The intersection code aims to calculate the
value of the $t$ parameter such that the point on ray's line is the point of
intersection on the primitive.

\subsection*{Cone}


\subsection*{Cylinder}

\subsection*{Disc}

\subsection*{Plane}

\subsection*{Torus}

