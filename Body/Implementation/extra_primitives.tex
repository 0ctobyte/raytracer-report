%*******************************************************
% Extra Primitives
%*******************************************************
\section{Extra Primitives}

This section will described how the \verb|intersection| method was
implemented for each of the extra primitives. As mentioned in the previous
section, the primtive classes were added to the \verb|primitives.cpp| source 
file.

The \verb|intersection| method takes two parameters. The first parameter is a
\verb|Ray| object and the second parameter is an \verb|Intersection| object. 

A ray can be represented by the equation $R = O + t*D$ where $O$ is the ray's 
origin and $D$ is the ray's direction. The \verb|Ray| object contains the ray's 
origin and direction. The point, $R$, on the ray can be determined given the $t$ 
parameter. The intersection code aims to calculate the value of the $t$ 
parameter such that the point on ray's line is the point of intersection on the 
primitive. 

The \verb|Intersection| object is where the intersection details are
stored such as the point of intersection, the normal at the point of
intersection (used for lighting calculations), the material of the primitive
(used for lighting calculations), the texture coordinates (used to index the
texture image), and the tangent vectors (used in the calculation for bump
mapping). All intersection tests are done in the primtives' local coordinate
system with unit sized primitives.

\subsection*{Cone}
The cone primitive is specified by the implicit equation $x^2 + y^2 = z^2$ and a
height, $h$. The height is by default set to $1.0$. The default orientation of 
the cone is along the z-axis. The implicit equation specifies an infinite double 
ended cone along the z-axis. The finite single-ended cone is specified by the
equation $x^2 + y^2 = z^2, -h < z < h$. 

In order to determine the intersection of a ray with a cone the implicit
equation must be solved by substituting in the equation of the ray:
\begin{equation}
  (O.x + t*D.x)^2 + (O.y + t*D.y)^2 = (O.z + t*D.z)^2\label{cone1}
\end{equation}
This equation expands to:
\begin{equation}
\begin{split}
  (D.x^2 + D.y^2 - D.z^2)*t^2 + \\
  (2*O.x*D.x + 2*O.y*D.y - 2*O.z*D.z)*t \\
  + (O.x^2 + O.y^2 - O.z^2) = 0\label{cone2}
\end{split}
\end{equation}
The equation specified in~(\ref{cone2}) can be solved for $t$ using the
quadratic equation. If the value of $t$ cannot be determined or the if $t$ is
negative, indicating that the intersection point is behind the ray's origin,
then there is no intersection. If $t$ is positive then the ray intersects the
cone and the intersection point is checked to make sure that the inequality $-h
< z < h$ is satisfied.

The texture coordinates for the cone primitive are calculated as follows:
\begin{equation}
  u = \frac{acos(Q \cdot \{1.0, 0.0, 0.0\})}{\pi}
\end{equation}

\subsection*{Cylinder}

\subsection*{Disc}

\subsection*{Plane}

\subsection*{Torus}

