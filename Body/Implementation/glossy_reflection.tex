%*******************************************************
% Glossy Reflections
%*******************************************************
\section{Glossy Reflections}

Glossy reflections can be enabled by setting the optional 
\newline \verb|<glossy_samples| parameter in the 
\verb|gr.render| Lua command.

Glossy reflections are implemented by perturbing the reflection ray to a random
point inside of a rectangle defined with a normal in the same direction as the
unperturbed reflection ray, $\vec{r}$ and centered at the point $r$ 
(Shirley \& Marschner, 2009). Many of these reflection rays are cast per pixel 
as defined by \verb|glossy_samples| parameter.

Thus, the following values must be calculated:
\begin{equation}
\begin{split}
  i = \frac{a}{2} + \zeta_{0}a \\
  j = \frac{a}{2} + \zeta_{1}a
\end{split}
\end{equation}
Where $a$ is the glossiness value and is calculated by $a = \frac{1}{1 + n_{s}}$
and $n_{s}$ is the Phong specular exponent. $\zeta_{0}$ and $\zeta_{1}$ are
uniform random numbers in the range $[0, 1]$.

These values are then used to calculate the perturbed reflection ray:
\begin{equation}
  \vec{r} = r + i\vec{U} + j\vec{V}
\end{equation}
Where $\vec{U}$ and $\vec{V}$ are orthogonal vectors in the plane of the
$a\times a$ rectangle centered at $r$. These vectors can be generated using the
algorithm described in (Hughes \& M{\"o}ller, 2005) where the normal vector is
simply in the direction of the unperturbed reflection ray.

