%*******************************************************
% Perlin Noise
%*******************************************************
\section{Perlin Noise}

Perlin noise can be enabled with the following Lua command:
\begin{lstlisting}
  set_perlin(<type>)
\end{lstlisting}
This sets the perlin texture on the \verb|PhongMaterial| for the specified 
\verb|GeometryNode|.

The Perlin noise algorithm is implemented exactly as described in
\cite{11_perlin_2002}. The noise function follows the reference implementation 
as closely as possibly.

The noise function takes in as input 3D coordinates and produces a floating
point value in the range $[0, 1]$. The algorithm requires that a hash table is
initialized with random variables in order to function properly. The
\verb|Perlin::init()| function is used to initialize this hash table at the
beginning of program execution.

The noise algorithm is performed by using the coordinates of the 3D point to 
retrieve values from the hash table which are in turn used to retrieve the 
gradient vectors of the eight cube corners of the unit integer lattice . The dot 
product of the gradient vectors are then taken with each of the eight unit cube 
corners specified by the given 3D point. These "influence" values are then
trilinearly interpolated using  the three splined interpolation values 
calculated using the fractional components of the 3D coordinates. 

The \verb|Perlin| class also implements three other static functions which
produce natural textures. The \verb|Perlin::marble| function produces marble
like textures. The \verb|Perlin::cloud| function produces cloud like textures
and the \verb|Perlin::wood| function produces wood like textures. These
functions implement algorithms found in \cite{7_kora_2007}. 

