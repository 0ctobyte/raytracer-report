%*******************************************************
% Phong Shading
%*******************************************************
\section{Phong Shading}

Phong shading of meshes is supported with the new Lua command:
\begin{lstlisting}
  gr.tri_mesh(<name>, <verts>, <faces>)
\end{lstlisting}
This command differs from the \verb|gr.mesh| command in that it subdivides the
faces into triangles and generates per-vertex normals on instantiation.

The following subsections describe the triangulation and normal generation
algorithms as well as the ray-triangle intersection and normal interpolation
algorithms. 

\subsection{Face Triangulation}
Triangulating the faces is a simple matter of looping through the faces and then
splitting the face into a triangle fan. The pseudocode of the algorithm is shown
below. A face is defined as holding an array of integers which are indices into
the vertex array.

\begin{lstlisting}[caption={Mesh face triangulation}]
  function triangulate(face_list):
    new_face_list = {}
    for each face in face_list:
      v0 = face[0];
      for i = 2 to face.size():
        new_face_list.append(face(v0, face[i-1], face[i]))
      end
    end
  end
\end{lstlisting}

\subsection{Per-Vertex Normal Generation}
Generating normals per-vertex is done by calculating the normals of each face
and then adding the normal to each of the vertices connecting the face. The
normals are then normalized to produce an averaged normal for each vertex. The
algorithm is shown in the below.

\begin{lstlisting}[caption={Generating per-vertex normals}]
  function generate_normals(face_list, vert_list):
    normal_list = {}
    for each face in face_list:
      normal = (face[1] - face[0]).cross(face[2] - face[0])
      for each index in face_list:
        normal_list[index] = normal_list[index] + normal
      end
    end

    for each normal in normal_list:
      normal.normalized()
    end
  end
\end{lstlisting}

\subsection{Ray-Triangle Intersection \& Normal Interpolation}
A triangle can be defined parametrically by:
\begin{equation}
\begin{split}
  f(u, v) &= wV_{0} + uV_{1} + vV_{2} \\
  w &= 1 - u - v
\end{split}
\end{equation}
The equation can be solved for $u$, $v$, and $w$ using the algorithm described
in \cite{8_möller_trumbore_1997}.

Using the parametric values of $u$, $v$, and $w$ a normal can be determined by
taking the weighted average of the surrounding vertex normals:
\begin{equation}
  \vec{N} = wN_{0} + uN_{1} + vN_{2}
\end{equation}

