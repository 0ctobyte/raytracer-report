%*******************************************************
% Refraction
%*******************************************************
\section{Refraction}

Refraction is supported with the optional \verb|<ni>| parameter in the
\newline \verb|gr.material| command. The parameter specifies the index 
of refraction of the material and is used in the refraction calculations.

Refraction is implemented similarly to mirror reflections. When a ray intersects
and object, a secondary refraction ray is generated and traced recursively.

I assume that the ray always goes from air to material or material to air. Thus,
it is necessary to first determine if the ray is entering an object or leaving
and object. This can be determined by taking the dot product of the ray's
direction vector, $\vec{D}$,  and the surface normal, $\vec{N}$. If the result 
is positive then the ray is leaving the object. Otherwise, the ray is entering 
the object. We can now define $n_{r} = \frac{n_{i}}{n_{t}}$ where $n_{i}$ is the 
index of refraction of the material that the ray is leaving and $n_{t}$ is the 
index of refraction of the material that the ray is entering. One of these 
indices of refraction will equal to $1.0$ (the index of refraction of air).

To calculate the refraction ray, $\vec{t}$, we must solve the following 
equation (de Greve, 2006):
\begin{equation}
\begin{split}
  \vec{t} = (-n_{r}(\vec{D}\cdot\vec\vec{N}) - \sqrt{1 - n_{r}^2(1 -
  (\vec{D}\cdot\vec{N})^2)})\vec{N} + n_{r}\vec{D}
\end{split}
\end{equation}
However, if the term inside the square root is negative then the refracted ray
undergoes total internal reflection and so the refraction ray will not be
traced.

The \verb|a4_refract| function in \verb|a4.cpp| performs the above calculation
as well calculating the Fresnel coefficient, $R$, using the following formulas:
\begin{equation}
\begin{split}
  a = \vec{D}\cdot\vec{N} \\
  b = 1 - n_{r}^2(1 - a^2) \\
  R_{0} = \frac{an_{i} - bn_{t}}{an_{i} + bn_{t}} \\
  R_{1} = \frac{an_{t} - bn_{i}}{an_{t} + bn_{i}} \\
  R = \frac{R_{0}^2 + R_{1}^2}{2}
\end{split}
\end{equation}
The Fresnel coefficient, $R$, is multiplied by the colour sampled by the
reflection ray and $(1 - R)$ is multiplied by the colour sampled by the
refraction ray to produce the final balanced colour of the secondary rays.

