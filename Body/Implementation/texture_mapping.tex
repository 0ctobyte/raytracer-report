%*******************************************************
% Texture Mapping
%*******************************************************
\section{Texture Mapping}

A texture mapping is enabled by setting a texture image to a
\verb|GeometryNode|. This can be with the following Lua command:
\begin{lstlisting}
  set_texture(<filename>)
\end{lstlisting}
This command sets the texture image on the \verb|PhongMaterial| attached to the
\verb|GeometryNode|. 

Texture mapping is implemented by generating $uv$ texture coordinates for each
of the primitives on an intersection. The texture coordinates are then used to
index into the texture map to produce a diffuse colour to be applied to the
intersection point in the lighting calculations. The following subsections
describe the $uv$ generation methods for each of the primitives and the texture
map sampling algorithm.

\subsection{Generating $uv$ Texture Coordinates}

The $uv$ texture coordinates are 2D coordinates with each coordinate having a
value in the range $[0, 1]$. This section will describe how these coordinates
are generated for each of the primitives.

\subsection*{Cone}
The texture coordinates for the cone primitive are calculated as follows:
\begin{equation}
  u = \frac{acos(Q \cdot \begin{bmatrix} 1.0 & 0.0 & 0.0 \end{bmatrix}^{T})}
  {\pi}
\end{equation}
\begin{equation}
  v = \frac{Q_{z}}{h}
\end{equation}
Where $Q$ is the intersection point on the cone and $h$ is the height of the
cone. Essentially, the $u$ coordinate is the angle of rotation of the vector 
(from the origin to the intersection point) about the $z$-axis divided by $\pi$ 
in order to scale it to the range $[0, 1]$. The $v$ coordinate is calculated by
dividing the $z$ coordinate of the intersection point by the height (since the
cone is aligned along the $z$-axis).

\subsubsectino*{Cylinder}
The texture coordinates for the cylinder primitive are calculated as follows

