%*******************************************************
% Soft Shadows
%*******************************************************
\section{Soft Shadows}

To enable soft shadows, the scene must contain area lights. The following Lua 
command creates an area light:
\begin{lstlisting}
  gr.disc_light(<position>, <colour>, <attenuation>, <normal>, <radius>)
\end{lstlisting}
  
An optional \verb|<shadow_samples>| parameter was added to the \verb|gr.render|
Lua commmand to support soft shadows. The parameter specifies the number of
shadow rays to be cast when performing lighting calculations on an intersection
point.

Soft shadows were implemented by randomly selecting a point on the disc light
and then casting a shadow ray from the intersection point to the point on the
disc light. The quality of the soft shadows increases as the number of shadow
rays that are cast increases.

The algorithm to select the random point on the disc light is implemented by
randomly generating a rotation angle and scaling factor. The rotation angle is
used to rotate a vector on the plane of the disc light and the scaling factor is 
used to scale the vector to a length between $[0, <radius>]$. Adding the 
resultant vector to the position of the disc light produces a point on the disc
light. The method to generate a vector coincident with a plane given the normal
of the plane is described in (Hughes & M{\"o}ller, 2005).

